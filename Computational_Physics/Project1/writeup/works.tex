\documentclass[11pt,a4paper,english]{article}
\usepackage{babel}
\usepackage{amssymb}
\usepackage{graphicx,subfigure}
\usepackage[export]{adjustbox}    % for positioning figure
\usepackage{textcomp}
\usepackage{fixltx2e}
\usepackage[usenames,dvipsnames,svgnames,table]{xcolor}

% some useful newcommands
\newcommand{\nl}{\nonumber \\}
\newcommand{\no}{\nonumber}
\newcommand{\ul}{\underline}
\newcommand{\ol}{\overline}

%some useful newcommands
\newcommand{\beq}{\begin{equation}}
\newcommand{\eeq}{\end{equation}}
\newcommand{\bfig}{\begin{figure}}
\newcommand{\efig}{\end{figure}}
\newcommand{\beqa}{\begin{eqnarray}}
\newcommand{\eeqa}{\end{eqnarray}}
\newcommand{\beqan}{\begin{eqnarray*}}
\newcommand{\eeqan}{\end{eqnarray*}}
\newcommand{\ba}{\begin{array}}
\newcommand{\ea}{\end{array}}
\newcommand{\ben}{\begin{enumerate}}
\newcommand{\een}{\end{enumerate}}
\newcommand{\bfl}{\begin{flushleft}}
\newcommand{\efl}{\end{flushleft}}
\newcommand{\btab}{\begin{tabular}}
\newcommand{\etab}{\end{tabular}}
\newcommand{\bit}{\begin{itemize}}
\newcommand{\eit}{\end{itemize}}
\newcommand{\bdes}{\begin{description}}
\newcommand{\edes}{\end{description}}
\newcommand{\bdm}{\begin{displaymath}}
\newcommand{\edm}{\end{displaymath}}
\newcommand {\IR} [1]{\textcolor{red}{#1}}

% for listing
\usepackage{enumitem}
\usepackage[ampersand]{easylist}
\ListProperties(Hide=100, Hang=true, Progressive=3ex, Style*=-- ,
Style2*=$\bullet$ ,Style3*=$\circ$ ,Style4*=\tiny$\blacksquare$ )    % for easylist
\newcommand{\begl}{\begin{easylist}}
\newcommand{\eegl}{\end{easylist}}

% for hyperlink
\usepackage{hyperref}             % for hyperlink
\hypersetup{
    colorlinks=true,
    linkcolor=blue,
    filecolor=magenta,      
    urlcolor=cyan,    
    bookmarks=true
    }


% Creating Title for the assessment

\title{Project 1}
\author{Bhishan Poudel}
\date{\today}

% to avoid indentation in paragraphs
\usepackage[parfill]{parskip}

% begin of document
\begin{document}
\maketitle
\tableofcontents
\listoffigures
\clearpage

%%%%%%%%%%%%%%%%%%%%%%%%%%%%%%%%%%%%%%%%%%%%%%%%%%%%%%%%%%%%%%%%%%%%%%%%%%%%%%%%%%%%%%%%%%%%%%%%%%%%%

\section{Question 1: Quantum Uncertainty in the Harmonic Oscillator}
In this question we studied the quantum uncertainty in the harmonic oscillator.\\

	
	\subsection{part a}
	
	In this part we studied the Hermite polynomials and harmonic oscillator wave functions.
	
	\subsubsection{Hermite Polynomials}
	
In this part I wrote a code to calculate Hermite polynomials. The data are saved for $n=1,2,3 $ for the plotting and data for $n=5,12$ are saved to compare exact values of table of Abramowitz.\\
The Hermite Polynomials were calculated using the recursion relation:\\

\beqa
H_{n}(x)=2xH_{n}(x)-2nH_{n-1}(x)
\eeqa

The first two Hermite Polynomials are:\\
$H_{0}(x)=1$\\
$H_{1}(x)=2x$

    folder       : qn1a/polynomial\\
	outputs      : n1.dat,n2.dat,n3.dat,n5.dat,n12.dat\\
	plots        : hnx123.eps\\
	
		%%%%% including figure %%%%%%%%%%%%%%%%%%
	\begin{figure}[h!]
	\centering
	\includegraphics [scale=0.6]{hnx123.eps}
	\caption{pr1qn1apoly }
	\end{figure}
	\clearpage
	%%%%%%%%%%%%%%%%%%%%%%%%%%%%%%%%%%%%%%%%

\subsubsection{Harmonic Oscillator Wave Functions}
	
In this part we studied the harmonic oscillator wave functions. The wave function of a spinless point particle in a quadratic potential well given by:\\
\beqa
\psi_{n}(x)=\frac{1}{\sqrt{2^{n}n!\sqrt{\pi}}} e^{\frac{-x^{2}}{2}} H_{n}(x)
\eeqa

I wrote a code to calculate wave functions for $n=0,1,2,3$ in the range $x=-4,4$\\

    folder       : qn1a/wavefunction\\
	outputs      : n0.dat,n1.dat,n2.dat,n3.dat\\
	plots        : pr1qn1a.eps\\
	
		%%%%% including figure %%%%%%%%%%%%%%%%%%
	\begin{figure}[h!]
	\centering
	\includegraphics [scale=0.6]{pr1qn1a.eps}
	\caption{hw6qn1a }
	\end{figure}
	\clearpage
	%%%%%%%%%%%%%%%%%%%%%%%%%%%%%%%%%%%%%%%%
	
	
	
	\subsection{part b}
In this part I plotted the wave function for $n=30$ from $x=-10,10$. I also calculated the time of run for the code using bash command time.

    folder       : qn1b\\
	outputs      : n30.dat\\
	plots        : pr1qn1b.eps\\
		
	%%%%% including figure %%%%%%%%%%%%%%%%%%
	\begin{figure}[h!]
	\centering
	\includegraphics [scale=0.6]{pr1qn1b.eps}
	\caption{hw6qn1b }
	\end{figure}
	\clearpage
	%%%%%%%%%%%%%%%%%%%%%%%%%%%%%%%%%%%%%%%%
	

	
	\subsection{part c}
In this part I wrote a code to calculate root mean square position. The root mean square position is given by:\\
\beq
<x^{2}> = \int_{-\infty}^{\infty} \!\! x^{2} |\psi_{n}(x)|^{2}  \,dx 
\eeq
Where $\psi_{n}(x)$ is given by eq.(2)\\

    folder       : qn1c\\
	source code  : gaulag.f90 (it was provided)\\
	source code  : pr1qn1c.f90\\
	outputs      : pr1qn1c.dat\\

\clearpage	
%%%%%%%%%%%%%%%%%%%%%%%%%%%%%%%%%%%%%%%%%%%%%%%%%%%%%%%%%%%%%%%%%%%%%%%%%%%%%%%%%%%%%%%%%
\section{Question 2: High Energy Scattering Cross Section}
In this problem we studied the high energy scattering of electron by alpha particle.

	
	\subsection{part a: Yukawa Potential}
In this part we studied the ionic potential:\\
The given values are:\\
$ Z=2 $\\
$ a_{0} = 0.5292 A^{0} $\\
$ r_{0}= a_{0}/4 = 0.1323 A^{0}$\\
$ \frac{e^{2}}{4\pi\epsilon_{0}} = 14.4 A^{0} eV $

	\beqa
	V(r) &=& \frac{1}{4\pi\epsilon_{0}} \frac{Ze^{2}}{r} e^{-r/r_{0}} \\
	 &=& Z \frac{e^{2}}{4\pi\epsilon_{0}} \frac{ e^{-r/r_{0}}}{r} \no\\
	 &=& 2 * 14.40 * \frac{ e^{-r/r_{0}}}{r}  \no\\
	 &=& 28.8 * \frac{ e^{-r/0.1323}}{r}  \no\\
	 &=& 28.8 * \frac{ e^{-7.559r}}{r} \quad\quad (eV) \no\\
	\eeqa

    folder       : qn2/potential\\
	source code  : pr1qn2pot.f90\\
	outputs      : pr1qn2pot.dat\\
	plots        : pr1qn2pot.eps\\
	
	
	%%%%% including figure %%%%%%%%%%%%%%%%%%
	\begin{figure}[h!]
	\centering
	\includegraphics [scale=0.6]{pr1qn2pot.eps}
	\caption{pr1qn2pot }
	\end{figure}
	\clearpage
	%%%%%%%%%%%%%%%%%%%%%%%%%%%%%%%%%%%%%%%%
	
	
	\subsection{part b: Scattering Amplitude}
In this question I have used following values:\\
Here, values used are:\\
$ mc^{2} = 0.5110 Mev $\\
$ k = 8 (A^{0})^{-1} $\\
$ \hslash c = 197.33 Mev fm $
Using Born approximation, the scattering amplitude is given by:\\
\beq
f(\theta)= -\frac{2m}{q\hslash^{2}}\!\! \int_0^\infty \!\! r V(r) sin(qr) \, dr
\eeq
Where, $ q=2ksin(\theta/2) $
where, k is initial or final magnitude of momentum. For low energy scattering $ kr_{0}<< 1$.
For high energy scattering I have chosen $ kr_{0} = 1$ then we get $ k = 8 (A^{0})^{-1} $.
\beqa
f(\theta) &=& -\frac{2m}{q\hslash^{2}}\!\! \int_0^\infty \!\! r V(r) sin(qr) \, dr \\
&=& -\frac{2mc^{2}}{(2ksin(\theta/2)\hslash^{2}c^{2}}\!\! \int_0^\infty \!\! r (28.80) \frac{ e^{-7.559r}}{r} sin(2krsin(\theta/2))  \, dr \no\\
&=& -\frac{0.4726}{sin(\theta/2)}\!\! \int_0^\infty \!\! sin(16rsin(\theta/2)) e^{-7.559r}\,dr  \no\\
\eeqa
Comparing to the standard format for generalized Gauss-Laguerre quadrature:\\
\bdm
I = \int_0^\infty \!\! e^{-r}r^{\alpha}\ \!f(r)\,dr 
\edm
we get:
$\alpha = 0 $ and \\
\bdm
f(r) = -\frac{0.4726}{sin(\theta/2)} sin(16rsin(\theta/2)) e^{r-7.559r}
\edm
    folder       : qn2/amplitude\\
	source code  : gaulag.f90   (obtained from Numerical Recipe)\\
	source code  : pr1qn2amp.f90\\
	outputs      : pr1qn2amp.dat\\
	plots        : pr1qn2amp.eps\\
	
	
	%%%%% including figure %%%%%%%%%%%%%%%%%%
	\begin{figure}[h!]
	\centering
	\includegraphics [scale=0.6]{pr1qn2amp.eps}
	\caption{pr1qn2amp }
	\end{figure}
	\clearpage
	%%%%%%%%%%%%%%%%%%%%%%%%%%%%%%%%%%%%%%%%


	
	\subsection{part c: Total Scattering Cross Section}
In this part I calculated the total cross section area.\\
The total cross section area is given by:\\
\bdm
\sigma=2\pi \quad \int_0^\pi \!\!  sin\theta |f(\theta)|^{2}\, d\theta
\edm 
Here i used the program for amplitude as a subroutine to find $f(\theta)$ and used Gauss-Legendre quadrature
to integrate the integral.\\
In the code i used do loop from $0$ to $20$ gauss points and 
obtained the converging result $0.4015E-01 (A^{0})^{2}$.\\
    folder       : qn2/cross-section\\
    source code  : gauleg.f90 (from Numerical Recipe) \\
	source code  : pr1qn2cross.f90\\
	outputs      : pr1qn2cross.dat\\ 

\end{document}

