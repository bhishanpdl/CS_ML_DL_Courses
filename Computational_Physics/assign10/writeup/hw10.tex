% Title : hw10
% Author: Bhishan Poudel
% Date  : Nov 11, 2015

\documentclass[11pt,a4paper,english]{article}
\usepackage{babel}
\usepackage{amsmath}
\usepackage{amssymb}
\usepackage{graphicx,subfigure}
\usepackage[export]{adjustbox}    % for positioning figure
\usepackage{textcomp}
\usepackage{fixltx2e}
\usepackage[usenames,dvipsnames,svgnames,table]{xcolor}

% some useful newcommands
\newcommand{\nl}{\nonumber \\}
\newcommand{\no}{\nonumber}
\newcommand{\ul}{\underline}
\newcommand{\ol}{\overline}

%some useful newcommands
\newcommand{\beq}{\begin{equation}}
\newcommand{\eeq}{\end{equation}}
\newcommand{\bfig}{\begin{figure}}
\newcommand{\efig}{\end{figure}}
\newcommand{\beqa}{\begin{eqnarray}}
\newcommand{\eeqa}{\end{eqnarray}}
\newcommand{\beqan}{\begin{eqnarray*}}
\newcommand{\eeqan}{\end{eqnarray*}}
\newcommand{\ba}{\begin{array}}
\newcommand{\ea}{\end{array}}
\newcommand{\ben}{\begin{enumerate}}
\newcommand{\een}{\end{enumerate}}
\newcommand{\bfl}{\begin{flushleft}}
\newcommand{\efl}{\end{flushleft}}
\newcommand{\btab}{\begin{tabular}}
\newcommand{\etab}{\end{tabular}}
\newcommand{\bit}{\begin{itemize}}
\newcommand{\eit}{\end{itemize}}
\newcommand{\bdes}{\begin{description}}
\newcommand{\edes}{\end{description}}
\newcommand{\bdm}{\begin{displaymath}}
\newcommand{\edm}{\end{displaymath}}
\newcommand {\IR} [1]{\textcolor{red}{#1}}

% for listing
\usepackage{enumitem}
\usepackage[ampersand]{easylist}
\ListProperties(Hide=100, Hang=true, Progressive=3ex, Style*=-- ,
Style2*=$\bullet$ ,Style3*=$\circ$ ,Style4*=\tiny$\blacksquare$ )    % for easylist
\newcommand{\begl}{\begin{easylist}}
\newcommand{\eegl}{\end{easylist}}

% for hyperlink
\usepackage{hyperref}             % for hyperlink
\hypersetup{
    colorlinks=true,
    linkcolor=blue,
    filecolor=magenta,      
    urlcolor=cyan,    
    bookmarks=true
    }


% Creating Title for the assessment

\title{Homework 10: Applications of the Metropolis Algorithm}
\author{Bhishan Poudel}
\date{Nov 11,2015}

% to avoid indentation in paragraphs
\usepackage[parfill]{parskip}

% begin of document
\begin{document}
\maketitle
\tableofcontents
\listoffigures
\clearpage

%%%%%%%%%%%%%%%%%%%%%%%%%%%%%%%%%%%%%%%%%%%%%%%%%%%%%%%%%%%%%%%%%%%%%%%%%%%%%%%%%%%%%%%%%%%%%%%%%%%%%

\section{Question 1: Metropolis Algorithm }
In this question I used metropolis algorithm to sample $2x$.\\
I proposed the values of $x$ from $0$ to $1$ using drand function.\\
The immediate next path values was taken in the range plus minus $0.1$ about current $x$.\\
We can see the xvalues lies between 0 and 1 in the file xvalues.dat.\\
I used 20 bins of 25 time steps and created 4 data segments. There is also
initial bin values and initial xvalues data file.
I chose 1000 walkers and nstep is 100.\\

I plotted the graph of sample points in the bins which is hw10qn1a.eps.\\
The animated pictures (gif files) of sample paths and xvalues was also created.\\
Executing the program hw10qn1.f90 gives output data files hw10qn1.dat and xvalues.dat.\\
Running xvalues.gp gives 4 png images and using website gifmaker.me I created the gif file xvalues.gif.
And I did similar for the samplePath.gif.

	The solution directory is :\\
	\begin{verbatim}
	location             : hw10/qn1/ 
	source code          : hw10qn1.f90
	datafiles            : hw10qn1.dat,initial.dat,xvalues.dat,initialbinvalues.dat
	gnuplot file         : xvalues.gp, samplePath.gp
	plots                : hw10qn1a.eps
	animated graphics    : samplePath.gif, xvalues.gif
	\end{verbatim}
 

	    The figures are shown below:\\
    %%%% including figure %%%%%%%%%%%%%%%%%%
	\begin{figure}[h!]
	\centering
	\includegraphics [scale=0.6]{figures/hw10qn1a.eps}
	\caption{sampling of $2x$ }
	\end{figure}
	\clearpage
	%%%%%%%%%%%%%%%%%%%%%%%%%%%%%%%%%%%%%%%		
%%%%%%%%%%%%%%%%%%%%%%%%%%%%%%%%%%%%%%%%%%%%%%%%%%%%%%%%%%%%%%%%%%%%%%%%%%%%%%%%%%%%%%%%%
\section{Question 2: Lattice Path Integration }

	In this part I solved the ground state probability for the $1D$ Harmonic Oscillator via
	Feynman path integration using the Metropolis algorith.\\
	Here, I translated the code qmc.java into hw10qn2.f90 and variables and paramter names are same
	as in the given book.\\
	I examined some of the actual space-time paths in the simulation.
	While comparing with classical trajectory, the path was found similar.
	Here in the code I chose max value is 250000.
	We can get a more precise value of wavefunction by taking this value larger.
	
	The plot shows wavefunction is gaussian and is similar to classical trajectory.	
		
		The solution directory is :\\
	\begin{verbatim}
	location             : hw10/qn2
	source code          : hw10qn2.f90
	datafiles            : hw10qn2a.dat,hw10qn2b.dat
	plots                : hw10qn2a.eps,hw10qn2b.eps	
	hints                : qmc.java (Landau 2E , chap 28.1.3) 
	\end{verbatim}

     The figures are shown below:\\
    %%%% including figure %%%%%%%%%%%%%%%%%%
	\begin{figure}[h!]
	\centering
	\includegraphics [scale=0.6]{figures/hw10qn2a.eps}
	\caption{ground state wavefunction }
	\end{figure}
	\clearpage
	%%%%%%%%%%%%%%%%%%%%%%%%%%%%%%%%%%%%%%%
	
		   
    %%%% including figure %%%%%%%%%%%%%%%%%%
	\begin{figure}[h!]
	\centering
	\includegraphics [scale=0.6]{figures/hw10qn2b.eps}
	\caption{position vs. time }
	\end{figure}
	\clearpage
	%%%%%%%%%%%%%%%%%%%%%%%%%%%%%%%%%%%%%%%		
    
%%%%%%%%%%%%%%%%%%%%%%%%%%%%%%%%%%%%%%%%%%%%%%%%%%%%%%%%%%%%%%%%%%%%%%%%%%%%%%%%%%%%%%%%%
\section{Question 3: Gravitational Potential }
    In this question I tested the wave function computation for the gravitational potential.
    Here, I modified the energy value of the code hw10qn2.f90 and got the appropriate energy value.
    I plotted the graph for wavefunction and position-time. Which can be seen below. 
    		The solution directory is :\\
	\begin{verbatim}
	location             : hw10/qn3
	source code          : hw10qn3.f90
	datafiles            : hw10qn3.dat
	\end{verbatim}
	
	    The figures are shown below:\\
    %%%% including figure %%%%%%%%%%%%%%%%%%
	\begin{figure}[h!]
	\centering
	\includegraphics [scale=0.6]{figures/hw10qn3a.eps}
	\caption{ground state wavefunction }
	\end{figure}
	\clearpage
	%%%%%%%%%%%%%%%%%%%%%%%%%%%%%%%%%%%%%%%
	
		    The figures are shown below:\\
    %%%% including figure %%%%%%%%%%%%%%%%%%
	\begin{figure}[h!]
	\centering
	\includegraphics [scale=0.6]{figures/hw10qn3b.eps}
	\caption{position vs. time }
	\end{figure}
	\clearpage
	%%%%%%%%%%%%%%%%%%%%%%%%%%%%%%%%%%%%%%%		

\end{document}

