% Title : hw11
% Author: Bhishan Poudel
% Date  : Nov 20, 2015

\documentclass[11pt,a4paper,english]{article}
\usepackage{babel}
\usepackage{amsmath}
\usepackage{amssymb}
\usepackage{graphicx,subfigure}
\usepackage[export]{adjustbox}    % for positioning figure
\usepackage{textcomp}
\usepackage{fixltx2e}
\usepackage[usenames,dvipsnames,svgnames,table]{xcolor}

% some useful newcommands
\newcommand{\nl}{\nonumber \\}
\newcommand{\no}{\nonumber}
\newcommand{\ul}{\underline}
\newcommand{\ol}{\overline}

%some useful newcommands
\newcommand{\beq}{\begin{equation}}
\newcommand{\eeq}{\end{equation}}
\newcommand{\bfig}{\begin{figure}}
\newcommand{\efig}{\end{figure}}
\newcommand{\beqa}{\begin{eqnarray}}
\newcommand{\eeqa}{\end{eqnarray}}
\newcommand{\beqan}{\begin{eqnarray*}}
\newcommand{\eeqan}{\end{eqnarray*}}
\newcommand{\ba}{\begin{array}}
\newcommand{\ea}{\end{array}}
\newcommand{\ben}{\begin{enumerate}}
\newcommand{\een}{\end{enumerate}}
\newcommand{\bfl}{\begin{flushleft}}
\newcommand{\efl}{\end{flushleft}}
\newcommand{\btab}{\begin{tabular}}
\newcommand{\etab}{\end{tabular}}
\newcommand{\bit}{\begin{itemize}}
\newcommand{\eit}{\end{itemize}}
\newcommand{\bdes}{\begin{description}}
\newcommand{\edes}{\end{description}}
\newcommand{\bdm}{\begin{displaymath}}
\newcommand{\edm}{\end{displaymath}}
\newcommand {\IR} [1]{\textcolor{red}{#1}}

% for listing
\usepackage{enumitem}
\usepackage[ampersand]{easylist}
\ListProperties(Hide=100, Hang=true, Progressive=3ex, Style*=-- ,
Style2*=$\bullet$ ,Style3*=$\circ$ ,Style4*=\tiny$\blacksquare$ )    % for easylist
\newcommand{\begl}{\begin{easylist}}
\newcommand{\eegl}{\end{easylist}}

% for hyperlink
\usepackage{hyperref}             % for hyperlink
\hypersetup{
    colorlinks=true,
    linkcolor=blue,
    filecolor=magenta,      
    urlcolor=cyan,    
    bookmarks=true
    }


% Creating Title for the assessment

\title{Homework 11:Differential Equations}
\author{Bhishan Poudel}
\date{Nov 20,2015}

% to avoid indentation in paragraphs
\usepackage[parfill]{parskip}

% begin of document
\begin{document}
\maketitle
\tableofcontents
\listoffigures
\clearpage

%%%%%%%%%%%%%%%%%%%%%%%%%%%%%%%%%%%%%%%%%%%%%%%%%%%%%%%%%%%%%%%%%%%%%%%%%%%%%%%%%%%%%%%%%%%%%%%%%%%%%

\section{Question 1: Solution of a Differential Equation }
    In this question I solved the differential equation:
    \beqa
    y'(x)= 2(y+1) \quad, -2<x<2 \quad, y(0) = 0
    \eeqa
    whose exact solution is :
    \beqa
    y(x) = e^{2x} - 1
    \eeqa
    using five different numerical methods to find the solution,viz.:
    \begin{easylist}
    & Euler's Method
    & Improved Euler's Method
    & Fourth Order Runge-Kutta Method
    & Adaptive Runge-Kutta Method
    & Picard's Method
    \end{easylist}
    And assessed their errors and robustness.
    \textbf{Note: in the plots $1b$ and $1d$ we may see that there is sudden transition of $y$ value from high value
            to low value this is not due to the mistake but it is due to the way I printed values in my datafile. In the
            data file i have printed from $0$ to $+2$ first, then $0$ to $-2$. If i take from $-2$ to $2$ the
            result will be same, nature of curve will remain same, error bar will remain same and there will
            be no sudden transion. } 
    

	\subsection{part a: Euler method with different step sizes}	
	In this part I used Euler' method to solve the differential equation.
	I used step-size $h=0.05,0.10,.015,0.20$ and plotted the results with error bar.\\
	From the graph we can say that when step size increases the error also increases.
	This means smaller value of step-size gives better result for the Euler's method.
	
		The solution directory is :\\
	\begin{verbatim}
	location             : hw11/qn1/qn1a
	source code          : hw11qn1a.f90 and euler.f90
	plots                : hw11qn1a.eps 
	datafiles            : euler05.dat, euler10.dat, euler15.dat, euler20.dat, also euler875.dat	  
	\end{verbatim}
			    The figures are shown below:\\
    %%%% including figure %%%%%%%%%%%%%%%%%%
	\begin{figure}[h!]
	\centering
	\includegraphics [scale=0.6]{figures/hw11qn1a.eps}
	\caption{Euler method with different step-sizes }
	\end{figure}
	\clearpage
	%%%%%%%%%%%%%%%%%%%%%%%%%%%%%%%%%%%%%%%  
	
	\subsection{part b: Comparison of Euler, Modified Euler, and fourth order Runge Kutta methods}	
	In this part I solved the given differential equation using three different methods, viz.
	 Euler, Improved Euler and Fourth Order Runge-Kutta methods. From the datafile and plot, 
	 we can see that Runge-Kutta method is slightly better than Improved-Euler-Method, and Improved
	 Euler method is much better than Euler method.
	
			The solution directory is :\\
	\begin{verbatim}
	location             : hw11/qn1/qn1b
	source code          : hw11qn1b.f90 
	plots                : hw11qn1b.eps
	datafiles            : hw11qn1b.dat
	provided codes       : rk4.f90 and test1rk4.f90 
	\end{verbatim}
			    The figures are shown below:\\
    %%%% including figure %%%%%%%%%%%%%%%%%%
	\begin{figure}[h!]
	\centering
	\includegraphics [scale=0.5]{figures/hw11qn1b.eps}
	\caption{Euler, Modified Euler, and fourth order Runge Kutta methods }
	\end{figure}
	\clearpage
	%%%%%%%%%%%%%%%%%%%%%%%%%%%%%%%%%%%%%%%  
	
	\subsection{part c: Adaptive Runge-Kutta Method}	
	In this part I used Adaptive Runge-Kutta Method ($difsis.f90$) to solve
	the given differential equation. I found that final value of $h$ was found to
	be $0.875$ in difsis, so I chose that value of step-size in Euler method (qn 1a)
	and compared the results. I found that adaptive Runge-Kutta method is much better
	than Euler's method. Also looking datafiles I found that Adaptive Runge-Kutta method
	is better than Fourth Order Runge Kutta and Improved-Euler method.
	
		The solution directory is :\\
	\begin{verbatim}
	location             : hw11/qn1/qn1c
	source code          : hw11qn1c.f90   and euler.f90 (inside qn1a)
	plots                : hw11qn1c.eps
	datafiles            : hw11qn1c.dat   and euler875.dat (inside qn1a)
	provided subroutines : difsis.f90 
	\end{verbatim}
			    The figures are shown below:\\
    %%%% including figure %%%%%%%%%%%%%%%%%%
	\begin{figure}[h!]
	\centering
	\includegraphics [scale=0.5]{figures/hw11qn1c.eps}
	\caption{Adaptive Runge-Kutta Method }
	\end{figure}
	\clearpage
	%%%%%%%%%%%%%%%%%%%%%%%%%%%%%%%%%%%%%%% 
	
	 
	\subsection{part d: Picard's Iteration Method}	
	In this part I used Picard's method to solve the given differential equation.
	For iterations $=4$, I found significant error. But for higher iterations
	There is less error and result is pretty accurate.
	
		The solution directory is :\\
	\begin{verbatim}
	location             : hw11/qn1/qn1d
	source code          : hw11qn1d.f90
	plots                : hw11qn1d.f90
	datafiles            : picard4.dat, picard8.dat, picard12.dat, picard16.dat 
	\end{verbatim}
	
		    The figures are shown below:\\
    %%%% including figure %%%%%%%%%%%%%%%%%%
	\begin{figure}[h!]
	\centering
	\includegraphics [scale=0.6]{figures/hw11qn1d.eps}
	\caption{Picard's Method with different iterations }
	\end{figure}
	\clearpage
	%%%%%%%%%%%%%%%%%%%%%%%%%%%%%%%%%%%%%%% 
		
%%%%%%%%%%%%%%%%%%%%%%%%%%%%%%%%%%%%%%%%%%%%%%%%%%%%%%%%%%%%%%%%%%%%%%%%%%%%%%%%%%%%%%%%%
\section{Question 2: The Ideal Harmonic Oscillator  }

	In this part I solved the Newton's equation of motion for the ideal harmonic
	oscillator:
	\beqa 
	\frac{d^{2}x}{dt^{2}} = -\frac{k}{m} x = - \omega_{0}^{2} x
	\eeqa
	with frequency
	\beqa 
	\omega_{0} = \frac{2\pi}{T} = \sqrt{\frac{k}{m}}
	\eeqa
	and analytic solution
	\beqa 
	x(t) = A sin(\omega_{0}t + \phi)
	\eeqa
	where, $A$ is amplitude and $\phi$ is phase constant.\\
	First I rewrite the second-order differential equation as two coupled first-
	order differential equations:
	\beqa 
	\frac{dx(t)}{dt} = v(t) \\
	\frac{dv(t)}{dt} = -\omega_{0}^{2}x(t) 
	\eeqa
	Then I used Runge-Kutta method to solve this system of coupled first order 
	differential equations.\\	
		
		The solution directory is :\\
	\begin{verbatim}
	location             : hw11/qn2    
	source code          : hw11qn2.f90
	datafiles            : positiontime05.dat, positiontime10.dat
	plots                : positiontime05.eps, positiontime10.eps
	datafiles            : energyerror05.dat,  energyerror10.dat   
	datafiles            : energyerror05a.dat, energyerror10a.dat 
	plots                : energyerror.eps,    energytime.eps,stability.eps
	provided subroutines : rk4.f90
	hints                : Landau 2E, Chapter 15 
	\end{verbatim}

	\subsection{part 2.1: }	
	Here, I picked values of $k$ and $m$ such that the period $T$ is a nice
	number to work with. I chose $T =1$.

	\subsection{part 2.2: }
	
    I tried out step sizes starting at $h = 0.10$ and then took smaller sizes ($h=0.05$).
    I solved for several periods. Here, the solution look smooth and have a period
    that never changes even after many oscillations.
    
			    The figures are shown below:\\
    %%%% including figure %%%%%%%%%%%%%%%%%%
	\begin{figure}[h!]
	\centering
	\includegraphics [scale=0.5]{figures/positiontime05.eps}
	\caption{position vs time plot at $h=0.05$ }
	\end{figure}
	%%%%%%%%%%%%%%%%%%%%%%%%%%%%%%%%%%%%%%% 
	
				    The figures are shown below:\\
    %%%% including figure %%%%%%%%%%%%%%%%%%
	\begin{figure}[h!]
	\centering
	\includegraphics [scale=0.5]{figures/positiontime10.eps}
	\caption{position vs time plot at $h=0.10$ }
	\end{figure}
	\clearpage
	%%%%%%%%%%%%%%%%%%%%%%%%%%%%%%%%%%%%%%%    	

    
    
    \subsection{part 2.3: }
	Here, I plotted the computed solution together with the analytic solution.
	I also compared the computed solutions with analytic one.
	I computed a relative error at $t = 9.5T, 19.5T, and \quad 29.5T$ for different step
	sizes as in part $2.2$.
    \begin{verbatim}
                          relative error at
    t      h=0.05          h=0.10
    9.5    0.169           0.266
    19.5   0.207           0.853
    29.5   0.559           0.936
    \end{verbatim}	
    		    The figures are shown below:\\
    %%%% including figure %%%%%%%%%%%%%%%%%%
	\begin{figure}[h!]
	\centering
	\includegraphics [scale=0.6]{figures/compare05.eps}
	\caption{ideal harmonic oscillator when $ h=0.05$ }
	\end{figure}
	%%%%%%%%%%%%%%%%%%%%%%%%%%%%%%%%%%%%%%% 

    %%%% including figure %%%%%%%%%%%%%%%%%%
	\begin{figure}[h!]
	\centering
	\includegraphics [scale=0.6]{figures/compare10.eps}
	\caption{ideal harmonic oscillator when $ h=0.10$ }
	\end{figure}
	\clearpage
	%%%%%%%%%%%%%%%%%%%%%%%%%%%%%%%%%%%%%%%     
	
	\subsection{part 2.4: }
	Here, we have not explicitly built energy conservation into the solution 
	of the differential equation. Nonetheless, since no friction is included, the 
	total energy must be a constant of motion. This is a demanding test of the 
	accuracy of the solution. Here, when $h=0.05$, I found constant value of
	energy to be $ 0.197389D+02 $.
    
    \subsection{part 2.5: }
    The total energy at any time $t$, given by
    \beqa 
    E = \frac{1}{2}kx(t)^{2} + \frac{1}{2}mv(t)^{2}
    \eeqa
    must be constant. I checked the numerically computed energy is constant at the 
    different times given when  $h=0.5$ and $h=0.10$ and plotted the relative error in percent as
    function of step size $h$. 
    		    The figures are shown below:\\
    %%%% including figure %%%%%%%%%%%%%%%%%%
	\begin{figure}[h!]
	\centering
	\includegraphics [scale=0.6]{figures/energytime.eps}
	\caption{Plot of energy vs time }
	\end{figure}

    %%%% including figure %%%%%%%%%%%%%%%%%%
	\begin{figure}[h!]
	\centering
	\includegraphics [scale=0.6]{figures/energyerror.eps}
	\caption{Plot of relative energy in percentage as a function of step-size }
	\end{figure}
	\clearpage
	%%%%%%%%%%%%%%%%%%%%%%%%%%%%%%%%%%%%%%% 

	\subsection{part 2.6: }
	The long-term \textit{stability} of the solution is given by\\
    \bdm
    \quad log[\frac{|E(t)-E(t=0)|}{E(t=0)}]
    \edm
    I plotted this stability versus time curve.
    		    The figures are shown below:\\
    %%%% including figure %%%%%%%%%%%%%%%%%%
	\begin{figure}[h!]
	\centering
	\includegraphics [scale=0.6]{figures/stability.eps}
	\caption{Plot of relative energy in percentage as a function of step-size }
	\end{figure}
	\clearpage
	%%%%%%%%%%%%%%%%%%%%%%%%%%%%%%%%%%%%%%% 
    
    
    \subsection{part 2.7: }
    Here, I added the viscous friction term to this model. I added a force\\
    \beqa 
    F_{f} = -bv
    \eeqa 
    where, $b$ is a parameter and $v$ is the velocity.\\
    I modified the code and investigated the qualitative changes of the solution that
    occur for increasing values of $b$:
    \begin{easylist}
    & \textbf{Underdamped:} $b \leq 2m\omega_{0} $
    & \textbf{Critically damped:} $b = 2m\omega_{0}$
    & \textbf{Over damped:} $b \geq 2m\omega_{0}$
    \end{easylist}
    Here, In my code I have taken $\omega_{0} = 2\pi \quad and \quad m =1$.
    
    I plotted the behavior of those three cases and demonstrated the behavior of
    the solution.
    
        		    The figures are shown below:\\
    %%%% including figure %%%%%%%%%%%%%%%%%%
	\begin{figure}[h!]
	\centering
	\includegraphics [scale=0.6]{figures/damped.eps}
	\caption{Plot of relative energy in percentage as a function of step-size }
	\end{figure}
	\clearpage
	%%%%%%%%%%%%%%%%%%%%%%%%%%%%%%%%%%%%%%% 
	
\end{document}

