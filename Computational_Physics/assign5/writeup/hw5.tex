\documentclass[11pt,a4paper,english]{article}
\usepackage{babel}
\usepackage{amssymb}
\usepackage{graphicx,subfigure}
\usepackage[export]{adjustbox}    % for positioning figure
\usepackage{textcomp}
\usepackage{fixltx2e}
\usepackage[usenames,dvipsnames,svgnames,table]{xcolor}

%some useful newcommands
\newcommand{\beq}{\begin{equation}}
\newcommand{\eeq}{\end{equation}}
\newcommand{\bfig}{\begin{figure}}
\newcommand{\efig}{\end{figure}}
\newcommand{\beqa}{\begin{eqnarray}}
\newcommand{\eeqa}{\end{eqnarray}}
\newcommand{\beqan}{\begin{eqnarray*}}
\newcommand{\eeqan}{\end{eqnarray*}}
\newcommand{\ba}{\begin{array}}
\newcommand{\ea}{\end{array}}
\newcommand{\ben}{\begin{enumerate}}
\newcommand{\een}{\end{enumerate}}
\newcommand{\bfl}{\begin{flushleft}}
\newcommand{\efl}{\end{flushleft}}
\newcommand{\btab}{\begin{tabular}}
\newcommand{\etab}{\end{tabular}}
\newcommand{\bit}{\begin{itemize}}
\newcommand{\eit}{\end{itemize}}
\newcommand{\bdes}{\begin{description}}
\newcommand{\edes}{\end{description}}
\newcommand{\bdm}{\begin{displaymath}}
\newcommand{\edm}{\end{displaymath}}
\newcommand {\IR} [1]{\textcolor{red}{#1}}

% for listing
\usepackage{enumitem}
\usepackage[ampersand]{easylist}
\ListProperties(Hide=100, Hang=true, Progressive=3ex, Style*=-- ,
Style2*=$\bullet$ ,Style3*=$\circ$ ,Style4*=\tiny$\blacksquare$ )    % for easylist
\newcommand{\begl}{\begin{easylist}}
\newcommand{\eegl}{\end{easylist}}

% for hyperlink
\usepackage{hyperref}             % for hyperlink
\hypersetup{
    colorlinks=true,
    linkcolor=blue,
    filecolor=magenta,      
    urlcolor=cyan,
    pdftitle={Sharelatex Example},
    bookmarks=true,
    pdfpagemode=FullScreen,
}


% Creating Title for the assessment

\title{Assignment 5}
\author{Bhishan Poudel}
\date{\today}

% begin of document
\begin{document}
\maketitle
\tableofcontents
\listoffigures
\clearpage

%%%%%%%%%%%%%%%%%%%%%%%%%%%%%%%%%%%%%%%%%%%%%%%%%%%%%%%%%%%%%%%%%%%%%%%%%%%%%%%%%%%%%%%%%%%%%%%%%%%%%

\section{Question 1: Integral of sine x}
In this problem we calculated the integral of $sin(x)$ from $0$ to pi.
The exact value is $2.0$.
We calculated numerical integral values using Trapezoid,Simpsons,and Gauss-Legendre method.
The template integ1.f90 was provided and further modification was done in the template.
errors are absolute errors. 

	
	\subsection{part a}
	
	In this part we prepared a table for the values of the given integral.\\
	source code is inside qn1: single and double folders. sp is single precision dp is double precision.\\
	source codes are: hw5qn1sp.f90 and hw5qn1dp.f90\\
	outputs  are    : hw5qn1sp.dat and hw5qn1dp.dat\\
	
	for the plot:\\
	source codes are: hw5qn1spplot.f90 and hw5qn1dpplot.f90\\
	outputs  are    : hw5qn1spplot.dat and hw5qn1dpplot.dat\\
	
	
	\subsection{part b}
    In this part, we plotted the graph of $log_{10}(e)$ vs. $log_{10}(N)$ 
    for 3 methods for single and double precisions.\\
	The graphs looks like this:\\
	
	%%%%% including figure %%%%%%%%%%%%%%%%%%
	\begin{figure}[h!]
	\centering
	\includegraphics [scale=0.6]{hw5qn1sp.eps}
	\caption{Plot for Qn1 single precision }
	\end{figure}
	\clearpage
	%%%%%%%%%%%%%%%%%%%%%%%%%%%%%%%%%%%%%%%%

%	%%%%% including figure %%%%%%%%%%%%%%%%%%
	\begin{figure}[h!]
	\centering
	\includegraphics [scale=0.6]{hw5qn1dp.eps}
	\caption{Plot for Qn1 double precision }
	\end{figure}
	\clearpage
	%%%%%%%%%%%%%%%%%%%%%%%%%%%%%%%%%%%%%%%%
	

	
	\subsection{part c}

	In part b, I have plotted the graph to determine the power-law dependence
	of the error on the number of points N.
	
	\subsection{part d}
	The plot for both single and double precision were drawn.
	
	\subsection{part e}
	Here, I used Trapezoid method to calculate the integral value of the given function.
	In this case we don't know the exact value of the integral.
	The integrals are calculated as the sum of the terms.
	We can choose some tolerance value such as:\\
	modulus(term/sum) is greater than tolerance.\\

\clearpage	
%%%%%%%%%%%%%%%%%%%%%%%%%%%%%%%%%%%%%%%%%%%%%%%%%%%%%%%%%%%%%%%%%%%%%%%%%%%%%%%%%%%%%%%%%
\section{Question 2: Integral of sine squared x}
The given integrals are:
\bdm
I = \int_{-1}^{1} \!\! \sqrt{1-x^2}\,dx 
\edm
and,\\

\bdm
I = \int_{-1}^{1} \!\! {sin^2 \theta}\,d\theta
\edm
The exact value $=\Pi/2 = 0.157079632679E+01$\\
I used Trapezoidal,Simpson,and Gauss-Legendre method to calculate the numerical interal
value of the given integrals separately.\\
I used double precision to solve the problem.\\
The source codes are  : hw5qn2a.f90 and hw5qn2b.f90\\
The outputs are       : hw5qn2a.dat,and hw5qn2b.dat\\
To get the precision $=0.157E+01$\\
for square root integral:\\
Trapezoid iteration $= 233$\\
Simpson iteration $= 17$\\
Gauss iteration $= 5$\\
To get the precision $=0.157079632679E+01$\\
for sine squared integral:\\
Trapezoid iteration $= 85$\\
Simpson iteration $=5$\\
Gauss iteration $=13$\\


\clearpage
\section{Question 3: Plank's black body radiation}

\subsection{part a: Gauss-Laguerre}
In this part we calculated numerical value using Gauss-Laguerre method for 2,4,6,8,and 10 points.
The error plot was also drawn.
The given inegral is :\\
\bdm
I = \int_0^\infty \!\! \frac{x^3}{e^{x} -1}\,dx
\edm
Comparing to the standard format for generalized Gauss-Laguerre quadrature:\\

\bdm
I = \int_0^\infty \!\! e^{-x}x^{\alpha}\ \!f(x)\,dx 
\edm
we get:
$\alpha = 3 $ and \\

\bdm
f(x) = \frac{1}{1-e^{-x}} 
\edm
The source code is: hw5qn3a.f90 \\
The plot   is     : hw5qn3.eps \\

	%%%%% including figure %%%%%%%%%%%%%%%%%%
	\begin{figure}[h!]
	\centering
	\includegraphics [scale=0.6]{hw5qn3.eps}
	\caption{Plot for Qn3 }
	\end{figure}
	\clearpage
	%%%%%%%%%%%%%%%%%%%%%%%%%%%%%%%%%%%%%%%%

\subsection{part b: Gauss-Legendre}
Here, to map the given interval (0 to infinity) into (0 to 1)\\
I substituted :
$x=\tan\frac{pi*y}{2}$ \\
Then, I calculated the value of f(y) for the integral of the form:\\
\bdm
I = \int_0^1 \! \!f(y)\,dy 
\edm
The source code is: hw5qn3b.f90 \\
The output   is     : hw5qn3b.dat \\
\clearpage
\section{Question 4 Sine Integral}
In this question the sample code integ3.f90 was provided for the integral:

\bdm
I = \int_1^{1000} \quad \frac{sin(x)}{x}\,dx
\edm
We have to solve numerically the integral:\\
\bdm
I = \int_1^{100} \quad \frac{sin(40x)}{x}\,dx
\edm
So, I substituted $y=40x$, then, I got the integral:\\
\bdm
I = \int_{40}^{4000} \quad \frac{sin(y)}{y}\,dy
\edm
The exact solution is : $Si(4000)-Si(40)$\\
Where Sine integral 'Si' is an entire function defined as:\\
\bdm
Si(z) = \int_{0}^{z} \quad \frac{sin(t)}{t}\,dt
\edm
The source code is  : hw5qn4.f90 and hw5qn4plot.f90 \\
The output      is  : hw5qn4.dat and hw5qn4plot.dat \\
The plot        is  : hw5qn4.eps \\

	%%%%% including figure %%%%%%%%%%%%%%%%%%
	\begin{figure}[h!]
	\centering
	\includegraphics [scale=0.6]{hw5qn4.eps}
	\caption{Plot for Qn4 }
	\end{figure}
	%%%%%%%%%%%%%%%%%%%%%%%%%%%%%%%%%%%%%%%%

\end{document}

