\documentclass[11pt,a4paper,english]{article}
\usepackage{babel}
\usepackage{amsmath}
\usepackage{amssymb}
\usepackage{graphicx,subfigure}
\usepackage[export]{adjustbox}    % for positioning figure
\usepackage{textcomp}
\usepackage{fixltx2e}
\usepackage[usenames,dvipsnames,svgnames,table]{xcolor}

% some useful newcommands
\newcommand{\nl}{\nonumber \\}
\newcommand{\no}{\nonumber}
\newcommand{\ul}{\underline}
\newcommand{\ol}{\overline}

%some useful newcommands
\newcommand{\beq}{\begin{equation}}
\newcommand{\eeq}{\end{equation}}
\newcommand{\bfig}{\begin{figure}}
\newcommand{\efig}{\end{figure}}
\newcommand{\beqa}{\begin{eqnarray}}
\newcommand{\eeqa}{\end{eqnarray}}
\newcommand{\beqan}{\begin{eqnarray*}}
\newcommand{\eeqan}{\end{eqnarray*}}
\newcommand{\ba}{\begin{array}}
\newcommand{\ea}{\end{array}}
\newcommand{\ben}{\begin{enumerate}}
\newcommand{\een}{\end{enumerate}}
\newcommand{\bfl}{\begin{flushleft}}
\newcommand{\efl}{\end{flushleft}}
\newcommand{\btab}{\begin{tabular}}
\newcommand{\etab}{\end{tabular}}
\newcommand{\bit}{\begin{itemize}}
\newcommand{\eit}{\end{itemize}}
\newcommand{\bdes}{\begin{description}}
\newcommand{\edes}{\end{description}}
\newcommand{\bdm}{\begin{displaymath}}
\newcommand{\edm}{\end{displaymath}}
\newcommand {\IR} [1]{\textcolor{red}{#1}}

% for listing
\usepackage{enumitem}
\usepackage[ampersand]{easylist}
\ListProperties(Hide=100, Hang=true, Progressive=3ex, Style*=-- ,
Style2*=$\bullet$ ,Style3*=$\circ$ ,Style4*=\tiny$\blacksquare$ )    % for easylist
\newcommand{\begl}{\begin{easylist}}
\newcommand{\eegl}{\end{easylist}}

% for hyperlink
\usepackage{hyperref}             % for hyperlink
\hypersetup{
    colorlinks=true,
    linkcolor=blue,
    filecolor=magenta,      
    urlcolor=cyan,    
    bookmarks=true
    }


% Creating Title for the assessment

\title{Homework 8: Randomness}
\author{Bhishan Poudel}
\date{Oct 24,2015}

% to avoid indentation in paragraphs
\usepackage[parfill]{parskip}

% begin of document
\begin{document}
\maketitle
\tableofcontents
\listoffigures
\clearpage

%%%%%%%%%%%%%%%%%%%%%%%%%%%%%%%%%%%%%%%%%%%%%%%%%%%%%%%%%%%%%%%%%%%%%%%%%%%%%%%%%%%%%%%%%%%%%%%%%%%%%

\section{Question 1: Random Sequences }
In this question I studied some programs to test and generate the random numbers.
The programs are mod(a,M), ran(flag), drand(flag), and a subroutine $sobseqn.f90$.\\

	
	\subsection{part abc: intrinsic function mod(a,p)}
	
	
	In this part I used the intrinsic program of fortran $90$ compiler called
	mod(a,p).\\
	Description:\\
    MOD(A,P) computes the remainder of the division of A by P.
    
    Arguments:

    A 	Shall be a scalar of type INTEGER or REAL.\\
    P 	Shall be a scalar of the same type and kind as A and not equal to zero.\\

    Return value:
    The return value is the result of A - (INT(A/P) * P). 
    The type and kind of the return value is the same as that of the arguments. 
    The returned value has the same sign as A and a magnitude less than the magnitude of P.
    
    In this question I varied A and P so that it gives different random numbers.\\
    Here, first argument $ A $ $=r_{i} = ar_{i-1}+c $\\
    second argument $ M = 256$\\
    I varied the values of $r(i)$ and fixed value of $m=256$ so that I got $256$ random numbers.\\
    Then I plotted $r(i)$ vs. $r(i+1)$.\\
      

	The solution directory is :\\
	\begin{verbatim}
	location             : hw8/qn1abc, qn1d and qn1e 
	source code          : hw8qn1abc.f90, hw8qn1d.f90, hw8qn1e.f90
	plots                : hw8qn1c.eps, hw8qn1d.eps, hw8qn1e.eps
	datafiles            : hw8qn1a.dat, hw8qn1c.dat, hw8qn1e.dat
	provided subroutines : sobseqn.f90 
	makefile             : Makefile
	\end{verbatim}
	
	    The figures are shown below:\\
    %%%% including figure %%%%%%%%%%%%%%%%%%
	\begin{figure}[h!]
	\centering
	\includegraphics [scale=0.6]{figures/hw8qn1c.eps}
	\caption{random numbers using mod(a,p) }
	\end{figure}
	\clearpage
	%%%%%%%%%%%%%%%%%%%%%%%%%%%%%%%%%%%%%%%	
	
	\subsection{part d: intrinsic function drand(seed) }
    In this part I studied the fortran intrinsic function drand(seed) to study
    random numbers. I created the data file for $r(i)$ vs. $r(i+1)$ and plotted
    the graph.\\
    
    	
	    The figures are shown below:\\
    %%%% including figure %%%%%%%%%%%%%%%%%%
	\begin{figure}[h!]
	\centering
	\includegraphics [scale=0.6]{figures/hw8qn1d.eps}
	\caption{random numbers using drand(seed) }
	\end{figure}
	\clearpage
	%%%%%%%%%%%%%%%%%%%%%%%%%%%%%%%%%%%%%%%	
	
	\subsection{part e: }
    In this part I used the subroutine `sobseqn' to create and study random numbers.
    Then, I plotted the graph of $r(i)$ vs. $r(i+1)$.
    
    	    The figures are shown below:\\
    %%%% including figure %%%%%%%%%%%%%%%%%%
	\begin{figure}[h!]
	\centering
	\includegraphics [scale=0.6]{figures/hw8qn1e.eps}
	\caption{random numbers using subroutine `sobseqn' }
	\end{figure}
	\clearpage
	%%%%%%%%%%%%%%%%%%%%%%%%%%%%%%%%%%%%%%%	
    
     

\clearpage	
%%%%%%%%%%%%%%%%%%%%%%%%%%%%%%%%%%%%%%%%%%%%%%%%%%%%%%%%%%%%%%%%%%%%%%%%%%%%%%%%%%%%%%%%%
\section{Question 2: Checks on Random Sequences }

	In this part I tested two random generator functions,
	viz. drand(seed) and `sobseqn' to check the uniformity of these functions.\\
	
		
		The solution directory is :\\
	\begin{verbatim}
	location             : hw8/qn2
	source code          : rand_check.f90, drand_check.f90
	datafiles            : rand_check.dat, drand_check.dat, hw8qn2b.dat
	provided subroutines : randcheck.f90,stest.f90, sobseqn.f90 
	\end{verbatim}

   	
	\subsection{part a: checking uniformity for drand }
	
	In this part I tested fortran built-in function ran and drand for the uniformity.
	The code $randcheck.f90$ was provided and I modified it.
	
	\subsection{part b: Testing sobol sequence }
	
    In this part I tested the given subroutine sobol sequence.
    The code $stest.f90$ was modified. The source code is $hw8qn2b.f90$.
    
    
%%%%%%%%%%%%%%%%%%%%%%%%%%%%%%%%%%%%%%%%%%%%%%%%%%%%%%%%%%%%%%%%%%%%%%%%%%%%%%%%%%%%%%%%%
\section{Question 3: Random Walk (Landau second edition page 147) }

   	
	\subsection{part ab: }
	
	In this part I modified the code $walk.f90$, normalized the plot and the plot looks like as
	I expected.
	
	    The figures are shown below:\\
    %%%% including figure %%%%%%%%%%%%%%%%%%
	\begin{figure}[h!]
	\centering
	\includegraphics [scale=0.6]{figures/hw8qn3ab.eps}
	\caption{plot of random walks }
	\end{figure}
	\clearpage
	%%%%%%%%%%%%%%%%%%%%%%%%%%%%%%%%%%%%%%% 
	
	\subsection{part cde: }
	
	In this part each trial have $1000$ steps and calculated the root mean square distance.
	I plotted $R_{rms}$ versus $\sqrt{N}$.
	I started $N$ with small value and I took $3$ significant figures.
	Here I took $N=1000$, when $N$ increases the gaussian distribution fits well and values of 
	rms distance and square root of $N$ becomes closer and closer.
	The plot is shown below:
	
		    The figures are shown below:\\
    %%%% including figure %%%%%%%%%%%%%%%%%%
	\begin{figure}[h!]
	\centering
	\includegraphics [scale=0.6]{figures/hw8qn3c.eps}
	\caption{plot of rms distance versus square root of steps }
	\end{figure}
	\clearpage
	%%%%%%%%%%%%%%%%%%%%%%%%%%%%%%%%%%%%%%% 
	
	
	\subsection{part f: }
	
    In this part I  plotted the scatterplot of random walk.
    The plot is uniform in all the four quadrant.
    The plot is shown below:
    
		    The figures are shown below:\\
    %%%% including figure %%%%%%%%%%%%%%%%%%
	\begin{figure}[h!]
	\centering
	\includegraphics [scale=0.6]{figures/hw8qn3f.eps}
	\caption{scatterplot }
	\end{figure}
	\clearpage
	%%%%%%%%%%%%%%%%%%%%%%%%%%%%%%%%%%%%%%% 
 
	
	

\end{document}

