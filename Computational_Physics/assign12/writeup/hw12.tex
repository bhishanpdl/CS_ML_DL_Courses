% Title : hw12
% Author: Bhishan Poudel
% Date  : Nov 28, 2015

\documentclass[11pt,a4paper,english]{article}
\usepackage{babel}
\usepackage{amsmath}
\usepackage{amssymb}
\usepackage{graphicx,subfigure}
\usepackage[export]{adjustbox}    % for positioning figure
\usepackage{textcomp}
\usepackage{fixltx2e}
\usepackage[usenames,dvipsnames,svgnames,table]{xcolor}

% some useful newcommands
\newcommand{\nl}{\nonumber \\}
\newcommand{\no}{\nonumber}
\newcommand{\ul}{\underline}
\newcommand{\ol}{\overline}

%some useful newcommands
\newcommand{\beq}{\begin{equation}}
\newcommand{\eeq}{\end{equation}}
\newcommand{\bfig}{\begin{figure}}
\newcommand{\efig}{\end{figure}}
\newcommand{\beqa}{\begin{eqnarray}}
\newcommand{\eeqa}{\end{eqnarray}}
\newcommand{\beqan}{\begin{eqnarray*}}
\newcommand{\eeqan}{\end{eqnarray*}}
\newcommand{\ba}{\begin{array}}
\newcommand{\ea}{\end{array}}
\newcommand{\ben}{\begin{enumerate}}
\newcommand{\een}{\end{enumerate}}
\newcommand{\bfl}{\begin{flushleft}}
\newcommand{\efl}{\end{flushleft}}
\newcommand{\btab}{\begin{tabular}}
\newcommand{\etab}{\end{tabular}}
\newcommand{\bit}{\begin{itemize}}
\newcommand{\eit}{\end{itemize}}
\newcommand{\bdes}{\begin{description}}
\newcommand{\edes}{\end{description}}
\newcommand{\bdm}{\begin{displaymath}}
\newcommand{\edm}{\end{displaymath}}
\newcommand {\IR} [1]{\textcolor{red}{#1}}

% for listing
\usepackage{enumitem}
\usepackage[ampersand]{easylist}
\ListProperties(Hide=100, Hang=true, Progressive=3ex, Style*=-- ,
Style2*=$\bullet$ ,Style3*=$\circ$ ,Style4*=\tiny$\blacksquare$ )    % for easylist
\newcommand{\begl}{\begin{easylist}}
\newcommand{\eegl}{\end{easylist}}

% for hyperlink
\usepackage{hyperref}             % for hyperlink
\hypersetup{
    colorlinks=true,
    linkcolor=blue,
    filecolor=magenta,      
    urlcolor=cyan,    
    bookmarks=true
    }


% Creating Title for the assessment

\title{Homework 12: Boundary Value Problems}
\author{Bhishan Poudel}
\date{Nov 28,2015}

% to avoid indentation in paragraphs
\usepackage[parfill]{parskip}

% begin of document
\begin{document}
\maketitle
\tableofcontents
\listoffigures
\clearpage

%%%%%%%%%%%%%%%%%%%%%%%%%%%%%%%%%%%%%%%%%%%%%%%%%%%%%%%%%%%%%%%%%%%%%%%%%%%%%%%%%%%%%%%%%%%%%%%%%%%%%

\section{Question 1: Legendre Polynomial}
In this question I calculated the values of Legendre polynomials of first kind 
of order three, four, and five.
I chose the range of x-values from $-0.9999$ to $0.9999$ and relative precision
$1e-6$. We can see in the data files that all three polynomials converges to $1.0000$
when $x=1$.
I also compared some of the values from my data and the Abramovich-Stegun Table (page $342$) for $n=3$.
We can also check other values in Wolfram Alpha, which are pretty much accurate.

\begin{verbatim}
x            0.250              0.500             0.750         1.000
table        -0.33 59 375       -0.43 75 000      -0.07 03 125  1.000
my value     -0.33 59 367       -0.43 74 972      -0.07 03 093  1.000  
\end{verbatim}
The table shows values are accurate upto third decimal points.
For the order four and five I used Wolfram Alpha to find the values.
command: legendre $p(n,x)$.

\begin{verbatim}
     x           my p(4,x)     wolfram p(4,x)        my p(5,x)     wolfram p(5,x)
     0.701001    -0.374935221  -0.41130010           -0.366720672  -0.366725162    
     0.801001    -0.229389047  -0.2300239            -0.398291064  -0.3982963965
     0.901001    0.159050061    0.213968570          -0.034653366  -0.0346544594
             

\end{verbatim}

	The solution directory is :\\
	\begin{verbatim}
	location             : hw12/qn1/ 
	source code          : hw12qn1.f90
	plots                : hw12qn1.eps
	datafiles            : n3.dat, n4.dat, n5.dat
	datafiles            : n3compare.dat, n4compare.dat, n5compare.dat (for comparison)
	provided subroutines : rk4.f90 
	\end{verbatim}
	
	    The figures are shown below:\\
    %%%% including figure %%%%%%%%%%%%%%%%%%
	\begin{figure}[h!]
	\centering
	\includegraphics [scale=0.6]{figures/hw12qn1.eps}
	\caption{Legendre Polynomials }
	\end{figure}
	\clearpage
	%%%%%%%%%%%%%%%%%%%%%%%%%%%%%%%%%%%%%%%	
	
\clearpage	
%%%%%%%%%%%%%%%%%%%%%%%%%%%%%%%%%%%%%%%%%%%%%%%%%%%%%%%%%%%%%%%%%%%%%%%%%%%%%%%%%%%%%%%%%
\section{Question 2: $3$D Isotropic Harmonic Oscillator  }

	The potential of $3$D harmonic oscillator is given by:\\
	
	\beq
	V(r) = \frac{1}{2}m\omega^2r^2 	
	\eeq
	Where, $m$ is mass of oscillator, $\omega$ is angular frequency and $r$ is
	radial distance.
	The energy of three dimensional oscillator is:
	\beq 
	E_n = (n + \frac{3}{2}) \hbar\omega
	\eeq
	The energy state $n$ is given by:\\
	\beq 
	n = 2k + l
	\eeq 
	Here, $k =$ no. of nodes.\\
	$l=$  angular momentum quantum number.\\
	$\hbar = \frac{h}{2\pi} = $ reduced planck's constant.
	
	The radial schrodinger equation is:\\
	\beq 
	-\frac{\hbar^2}{2m}u'' + [V + \frac{\hbar^2}{2m} \frac{l(l+1)}{r^2}]u = Eu
	\eeq 
	Rearranging yields:\\
	\beq 
	u'' = [\frac{l(l+1)}{r^2} - \frac{2m}{\hbar^2}(E-V) ]u 
    \eeq 

   	
	\subsection{part 2.1: Solving radial equation }
	
	In this part I solved the radial differential equation using a subroutine from the internet mentioned below.
	I calculated the ground state energy and its value is:\\
	\begin{verbatim}
	        for l=0  for l=1   for l=2
	 Energy 5.499    6.499     7.499   
	 
	\end{verbatim}
	
	
		The solution directory is :\\
	\begin{verbatim}
	location             : hw12/qn2/ 
	source code          : hw12qn2.f90
	plots                : hw12qn2.eps, hw12qn2d.eps
	datafiles            : hw12qn2.dat, hw12qn2d.dat
	downloded subroutine : nsolve.f90 and ho.f90
	reference            : http://infty.net/nsolve/nsolve.html
	\end{verbatim}
	
		    The figures are shown below:\\
    %%%% including figure %%%%%%%%%%%%%%%%%%
	\begin{figure}[h!]
	\centering
	\includegraphics [scale=0.6]{figures/hw12qn2.eps}
	\caption{$3$d harmonic oscillator }
	\end{figure}
	\clearpage
	%%%%%%%%%%%%%%%%%%%%%%%%%%%%%%%%%%%%%%%    
    
    \subsection{part 2.4: Perturbed $3d$ harmonic oscillator }
	
    In this part I added a quartic perturbation term $\lambda\rho^4$ to the 
    potential of the oscillator and calculated the ground state energy and wavefunction
    for $l=0$. I chose $\lambda=0.1$.\\
    The value of perturbed ground state energy is :\\
    $E_{0(perturbed)} = 7.899 $\\
    The plot of wavefunction is showen below:\\
        %		    The figures are shown below:\\
    %%%% including figure %%%%%%%%%%%%%%%%%%
	\begin{figure}[h!]
	\centering
	\includegraphics [scale=0.6]{figures/hw12qn2d.eps}
	\caption{Perturbed $3d$ harmonic oscillator }
	\end{figure}
	\clearpage
	%%%%%%%%%%%%%%%%%%%%%%%%%%%%%%%%%%%%%%%	
    
    

\end{document}

