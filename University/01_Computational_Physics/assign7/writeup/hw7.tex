\documentclass[11pt,a4paper,english]{article}
\usepackage{babel}
\usepackage{amsmath}
\usepackage{amssymb}
\usepackage{graphicx,subfigure}
\usepackage[export]{adjustbox}    % for positioning figure
\usepackage{textcomp}
\usepackage{fixltx2e}
\usepackage[usenames,dvipsnames,svgnames,table]{xcolor}

% some useful newcommands
\newcommand{\nl}{\nonumber \\}
\newcommand{\no}{\nonumber}
\newcommand{\ul}{\underline}
\newcommand{\ol}{\overline}

%some useful newcommands
\newcommand{\beq}{\begin{equation}}
\newcommand{\eeq}{\end{equation}}
\newcommand{\bfig}{\begin{figure}}
\newcommand{\efig}{\end{figure}}
\newcommand{\beqa}{\begin{eqnarray}}
\newcommand{\eeqa}{\end{eqnarray}}
\newcommand{\beqan}{\begin{eqnarray*}}
\newcommand{\eeqan}{\end{eqnarray*}}
\newcommand{\ba}{\begin{array}}
\newcommand{\ea}{\end{array}}
\newcommand{\ben}{\begin{enumerate}}
\newcommand{\een}{\end{enumerate}}
\newcommand{\bfl}{\begin{flushleft}}
\newcommand{\efl}{\end{flushleft}}
\newcommand{\btab}{\begin{tabular}}
\newcommand{\etab}{\end{tabular}}
\newcommand{\bit}{\begin{itemize}}
\newcommand{\eit}{\end{itemize}}
\newcommand{\bdes}{\begin{description}}
\newcommand{\edes}{\end{description}}
\newcommand{\bdm}{\begin{displaymath}}
\newcommand{\edm}{\end{displaymath}}
\newcommand {\IR} [1]{\textcolor{red}{#1}}

% for listing
\usepackage{enumitem}
\usepackage[ampersand]{easylist}
\ListProperties(Hide=100, Hang=true, Progressive=3ex, Style*=-- ,
Style2*=$\bullet$ ,Style3*=$\circ$ ,Style4*=\tiny$\blacksquare$ )    % for easylist
\newcommand{\begl}{\begin{easylist}}
\newcommand{\eegl}{\end{easylist}}

% for hyperlink
\usepackage{hyperref}             % for hyperlink
\hypersetup{
    colorlinks=true,
    linkcolor=blue,
    filecolor=magenta,      
    urlcolor=cyan,    
    bookmarks=true
    }


% Creating Title for the assessment

\title{Homework 7: Eigensystems}
\author{Bhishan Poudel}
\date{\today}

% to avoid indentation in paragraphs
\usepackage[parfill]{parskip}

% begin of document
\begin{document}
\maketitle
\tableofcontents
\listoffigures
\clearpage

%%%%%%%%%%%%%%%%%%%%%%%%%%%%%%%%%%%%%%%%%%%%%%%%%%%%%%%%%%%%%%%%%%%%%%%%%%%%%%%%%%%%%%%%%%%%%%%%%%%%%

\section{Question 1: Testing Matrix Calls}
In this question we studied the quantum uncertainty in the harmonic oscillator.\\

	
	\subsection{part a: Verify the inverse of the matrix}
	
	In this part I verified the inverse of the matrix A .
	First i solved the matrix analytically.
	Then I wrote the program to calculate the inverse of the given matrix.\\
	The given matrix is :\\
    \begin{displaymath}
      A=
  		\begin{pmatrix}
    			4 & -2 & 1 \\
    			3 &  6 & -4 \\
    			2 &  1 &  8 
  		\end{pmatrix}
	\end{displaymath}
	The inverse of matrix $A$ is:\\
	\bdm
	 A^{-1} = \frac{1}{263} 	      
  		\begin{pmatrix}
    			52 & 17 & 2 \\
    		   -32 & 30 & 19 \\
    		   -9  & -8 & 30 
  		\end{pmatrix}
	\edm
	The solution location is :\\
	\begin{verbatim}
	location             : hw7/qn1abc
	provided subroutine  : dminv.f90
	source code          : dminv_test.f90
	additional subroutine: print_matrix.f90
	makefile             : Makefile
	\end{verbatim}
	
	\subsection{part b: verify properties of inverse matrix}
	
    In this part I verified the inverse matrix property. The source code is same as in part $a$.
    I got unit matrix of dimension three for single significant figures precision, however, for other 
    significant figures I got some deviations from perfect anticipated unit matrix.

   \subsection{part c: comparison of determinant}
   In this part I compared the determinant from my analytic and computational result. They matched
   correctly. Source code is same as in part a.	
	
	
	\subsection{part d: use of LAPACK subroutine $dgesv$ }
    In this part I used the $LAPACK$ library routine $ dgesv$ to solve the system of three linear equations
    of the form:
    
    \begin{eqnarray}
     AX=B
     \end{eqnarray}
      
	Where:\\
	    \begin{displaymath}
      A=
  		\begin{pmatrix}
    			4 & -2 & 1 \\
    			3 &  6 & -4 \\
    			2 &  1 &  8 
  		\end{pmatrix}
	\end{displaymath}
	And,
	    \begin{displaymath}
      B=
  		\begin{pmatrix}
    			4 \\
    		  -10 \\
    		   22 
  		\end{pmatrix}
	\end{displaymath}
	I solved this equation for $X$.\\
	The solution is :\\
	\begin{displaymath}
      X=
  		\begin{pmatrix}
    			0.312 \\
    		  -0.038 \\
    		   2.677 
  		\end{pmatrix}
	\end{displaymath}
	\begin{verbatim}
	location             : hw7/qn1d/
	library used         : LAPACK
	source code          : dgesv_test.f90
	makefile             : Makefile
	\end{verbatim}
	
	\subsection{part e: use of LAPACK subroutines dsyev and dgeev}
    In this part the given matrix is symmetric. At first i used the lapack routine for symmetric
    matrix 'dsyev' to calculate the eigenvalues and eigenvectors.\\
    Then, i used the another lapack routine 'dgeev' to calculate the real eigenvalues and right 
    eigenvectors.I got the same eigenvalues and proportionate eigenvectors. 
    Note that these lapack subroutine gives normalized eigenvectors and eigenvectors 
    are different from that in Wolfram Alpha website.
    	The given symmetric matrix is :\\
    \begin{displaymath}
      A=
  		\begin{pmatrix}
    			1 & -4 &  2 \\
    		   -4 &  1 & -2 \\
    			2 & -2 & -2 
  		\end{pmatrix}
	\end{displaymath}
	Here, I found same eigenvalues for both of the subroutines.\\
	The eigenvector for non-degenerate eigenvalue is same.\\
	However, for the doubly degenerate eigenvalue, the eigenvectors are different.\\   
    The solution location is :\\
	\begin{verbatim}
	location             : hw7/qn1e/dgesv and dgeev
	library used         : LAPACK
	source code          : dgesv_test.f90, dgeev_test.f90
	output files         : dgesv.dat,dgeev.dat
	makefile             : Makefile
	\end{verbatim}
        
    \subsection{part f: use of LAPACK subroutine dgeev}
    In this part the given input matrix is non-symmetric. I used the LAPACK subroutine dgeev to find the
    eigenvalues and eigenvectors. The eigenvectors are printed as columns in the same order as 
    the eigenvalues appears in my code.
        	The given non-symmetric matrix is :\\
    \begin{displaymath}
      A=
  		\begin{pmatrix}
    			-2 &  2 &  -3 \\
    		     2 &  1 &  -6 \\
    			-1 & -2 &   0 
  		\end{pmatrix}
	\end{displaymath}
	Here, I found that the calculated eigenvectors are proportional to exact eigenvectors.\\
	Here, for the degenerate eigenvalue -3, the eigen vectors are:\\
    $ X_{2}^{'}	= 0.23 X_{2} - 0.11 X_{3} $\\
    $ X_{3}^{'}	= 0.67 X_{2} - 0.48 X_{3} $\\
	
	
	    
    The solution location is :\\
    \begin{verbatim}
	location             : hw7/qn1f/
	library used         : LAPACK
	source code          : dgeev_test.f90
	output files         : hw7qn1f.dat
	makefile             : Makefile
	\end{verbatim}

\clearpage	
%%%%%%%%%%%%%%%%%%%%%%%%%%%%%%%%%%%%%%%%%%%%%%%%%%%%%%%%%%%%%%%%%%%%%%%%%%%%%%%%%%%%%%%%%
\section{Question 2: Schrodinger equation via diagonalization }

   	
	\subsection{part a: Derivation of radial wavefunction}
	
	In this part the radial part of wave function was provided as equation 10.\\
	I solved the further steps and derived the equation 15.\\
	The file is $hw7qn2.pdf$.
	
	\subsection{part b: solving radial equation using diagonalizing routine dsyev}
	
    In this part I solved the radial part of schrodinger equation using diagonalization method.
    I followed the process of Morten's Computational Physics $2014$ Fall notes section $7.8$.
    The given potential of the harmonic oscillator is :\\
    \beqa
    V(x)=\frac{1}{2}kx^{2}
    \text{  where, k=1}
    \eeqa
    The step size is given by:\\
    \beqa
    h&=&\frac{ R_{max} - R_{min} } {N_{step}}\\
    &=&\frac{ 10 - (-10) } {100}\no\\
    &=& 0.1\no   
    \eeqa
    Then i created arrays of dimension $0$ to nstep for $x_{i}$ and $V_{i}$.\\
    where,\\
    \beqa
    x(i) &=& R_{min} + i * h\\
    \text{ where, i = 1,2,3,...,nstep-1}\no\\
    V(i)&=& x(i) * x(i)    
    \eeqa
    The diagonal elements d(i) has dimension $1$ to $nstep-1$.\\
    Then, the diagonal matrix elements are given by:\\
    \beqa
    d(i)=\frac{2}{h^{2}} + V(i)
    \eeqa
    The non-vanishing non-diagonal elements e(i) has dimension $1$ to $nstep-1$.\\
    The all non-vanishing non-diagonal matrix elements are given by:\\
    \beqa
    e(i)=\frac{-1}{h^{2}}
    \eeqa
    And, all the rest elements are zero.\\
    Then i constructed the symmetric matrix $A$ such that:\\
    \bdm
    Au = 2E u
    \edm
    Here, $A$ is symmetric matrix of dimension (nstep * nstep).\\
    $u$ is column eigenvector of dimension ($1$ to $nstep-1$).\\
    The matrix $A$ looks like this:\\
    \begin{displaymath}
    A = 
 	\begin{pmatrix}
 	 	d_{1}   & e_{1}   & 0         & 0      & \cdots   & 0            & 0 \\
	 	e_{1}   & d_{2}   & e_{}2     & 0      & \cdots   & 0            & 0 \\
	 	0       & e_{2}   & d_{3}     & e_{3}  & 0        & \cdots       & 0 \\
	 	\cdots  & \cdots  & \cdots    & \cdots &\cdots    & \cdots       & \cdots  \\
	 	0       & \cdots  & \cdots    & \cdots & \cdots   & d_{nstep-2}  & e_{nstep-1} \\
	 	0       & \cdots  &\cdots     & \cdots & \cdots   & e_{nstep-1}  & d_{nstep-1} 
 	\end{pmatrix}
    \end{displaymath}
        The matrix $u$ looks like this:\\
    \begin{displaymath}
    u = 
 	\begin{pmatrix}
 	 	u_{1}  \\
	 	u_{2}  \\
	 	\cdots \\
	 	\cdots \\
	 	\cdots \\
	 	u_{nstep-1}  
 	\end{pmatrix}
    \end{displaymath}        
    
    Diagonalizing the matrix will yield the twice of the energy values.\\
    For three dimensional harmonic oscillator the energy eigenvalues are given by:\\
    \beqa
    E_{n,l}&=& (2k+l+\frac{3}{2}) \hbar \omega\\
    n&=& 2k+l
    \eeqa
    where k is no. of nodes in a wavefunction.\\
    Here, I found $k$ from the plot of wavefunction which was zero, one and two for ground state, first
    excited, and second excited state respectively.\\
    Also, here $l=0$ for all the cases.\\
    Hence,\\
    \clearpage
    ground state $$ E_{0} = 3/2 \quad \hbar \omega $$ \\
    first excited state $$ E_{1} = 7/2 \quad \hbar \omega $$ \\
    second excited state $$ E_{2} = 11/2 \quad \hbar \omega $$ \\
    
    For simplicity, i have plotted the graph for one dimensional case.\\
    I have plotted eigenvalue versus value of h.\\
    The nature of the plot will be same and energy values will be changed.\\
    Here, each of my energy eigenvalue are in terms of $ \frac{\hbar\omega}{2}$.\\
    Also, no. of nodes are 1,2,3 for ground,first,and second excited states.\\
    for ground state:\\
    1d value $=  1 $ so,  3d value $= 1 + (1/2) = 3/2$ \\
    for first excited state:\\
    1d value $=  3 $ so,  3d value $= 2 + (3/2) = 7/2$ \\
    for second excited state:\\
    1d value $=  5 $ so,  3d value $= 3 + (5/2) =11/2$ \\
    
    
    The figures are shown below:\\
    %%%%% including figure %%%%%%%%%%%%%%%%%%
	\begin{figure}[h!]
	\centering
	\includegraphics [scale=0.6]{e0.eps}
	\caption{Energy eigenvalue for ground state }
	\end{figure}
	\clearpage
	%%%%%%%%%%%%%%%%%%%%%%%%%%%%%%%%%%%%%%%%
	
	%%%%% including figure %%%%%%%%%%%%%%%%%%
	\begin{figure}[h!]
	\centering
	\includegraphics [scale=0.6]{e1.eps}
	\caption{Energy eigenvalue for first excited state }
	\end{figure}
	\clearpage
	%%%%%%%%%%%%%%%%%%%%%%%%%%%%%%%%%%%%%%%%
	
	%%%%% including figure %%%%%%%%%%%%%%%%%%
	\begin{figure}[h!]
	\centering
	\includegraphics [scale=0.6]{e2.eps}
	\caption{Energy eigenvalue for second excited state }
	\end{figure}
	\clearpage
	%%%%%%%%%%%%%%%%%%%%%%%%%%%%%%%%%%%%%%%%
    
   \subsection{part c: extrapolation to get energy value when h tends to zero}
	
	Here, I used xmgrace to extrapolate the energy value.\\
	The values found are 1,3,and 5.\\
	The plots are shown above.\\	
	\subsection{part d: plot of radial wavefunction }
    In this part I plotted the first three eigenvectors from the code `hw7qn2.f90'.\\
    %%%%% including figure %%%%%%%%%%%%%%%%%%
	\begin{figure}[h!]
	\centering
	\includegraphics [scale=0.6]{u123.eps}
	\caption{Radial wave functions for three lowest eigenstates }
	\end{figure}
	\clearpage
	%%%%%%%%%%%%%%%%%%%%%%%%%%%%%%%%%%%%%%%%
    The solution location is :\\
	\begin{verbatim}
	location             : hw7/qn2
	library              : LAPACK (dsyev subroutine)
	source code          : hw7qn2.f90
	output data file     : u123.dat
	plot                 : u123.eps
	makefile             : Makefile
	handsout             : hw7qn2a.pdf
	\end{verbatim} 

\end{document}

