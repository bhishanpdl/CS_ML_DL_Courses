\documentclass[11pt,a4paper,english]{article}
\usepackage{babel}
\usepackage{amssymb}
\usepackage{graphicx,subfigure}
\usepackage[export]{adjustbox}    % for positioning figure
\usepackage{textcomp}
\usepackage{fixltx2e}
\usepackage[usenames,dvipsnames,svgnames,table]{xcolor}

%some useful newcommands
\newcommand{\beq}{\begin{equation}}
\newcommand{\eeq}{\end{equation}}
\newcommand{\bfig}{\begin{figure}}
\newcommand{\efig}{\end{figure}}
\newcommand{\beqa}{\begin{eqnarray}}
\newcommand{\eeqa}{\end{eqnarray}}
\newcommand{\beqan}{\begin{eqnarray*}}
\newcommand{\eeqan}{\end{eqnarray*}}
\newcommand{\ba}{\begin{array}}
\newcommand{\ea}{\end{array}}
\newcommand{\ben}{\begin{enumerate}}
\newcommand{\een}{\end{enumerate}}
\newcommand{\bfl}{\begin{flushleft}}
\newcommand{\efl}{\end{flushleft}}
\newcommand{\btab}{\begin{tabular}}
\newcommand{\etab}{\end{tabular}}
\newcommand{\bit}{\begin{itemize}}
\newcommand{\eit}{\end{itemize}}
\newcommand{\bdes}{\begin{description}}
\newcommand{\edes}{\end{description}}
\newcommand{\bdm}{\begin{displaymath}}
\newcommand{\edm}{\end{displaymath}}
\newcommand {\IR} [1]{\textcolor{red}{#1}}

% for listing
\usepackage{enumitem}
\usepackage[ampersand]{easylist}
\ListProperties(Hide=100, Hang=true, Progressive=3ex, Style*=-- ,
Style2*=$\bullet$ ,Style3*=$\circ$ ,Style4*=\tiny$\blacksquare$ )    % for easylist
\newcommand{\begl}{\begin{easylist}}
\newcommand{\eegl}{\end{easylist}}

% for hyperlink
\usepackage{hyperref}             % for hyperlink
\hypersetup{
    colorlinks=true,
    linkcolor=blue,
    filecolor=magenta,      
    urlcolor=cyan,
    pdftitle={Sharelatex Example},
    bookmarks=true,
    pdfpagemode=FullScreen,
}


% Creating Title for the assessment

\title{Assignment 4}
\author{Bhishan Poudel}
\date{\today}

% begin of document
\begin{document}
\maketitle
\tableofcontents
\listoffigures
\clearpage

%%%%%%%%%%%%%%%%%%%%%%%%%%%%%%%%%%%%%%%%%%%%%%%%%%%%%%%%%%%%%%%%%%%%%%%%%%%%%%%%%%%%%%%%%%%%%%%%%%%%%

\section{Question 1}

	This program computes the first derivative of functions\\
	\beqa
	f(x)&=&cos(x)\\
	f(x)&=&exp(x)\\
	f(x)&=&\sqrt(x)\\
	\eeqa
	at points $x=0.1$,$x=1$,$x=30$ \\
	
	\subsection{part a}
	
	
	The source code is:\\
	assign4/qn1/hw4qn1.f90\\
	
	The outputs are :\\
	
	\begin{verbatim}
	for single precision:
	cos01sp.dat,cos1sp.dat,cos30sp.dat for x=0.1,1.0,and30.0 respectively
	exp01sp.dat,exp1sp.dat,exp30sp.dat
	sq01sp.dat,sq1sp.dat,sq30sp.dat
	
	For double precision:
	cos01dp.dat,cos1dp.dat,cos30dp.dat for x=0.1,1.0,and30.0 respectively
	sq01dp.dat,sq1dp.dat,sq30dp.dat
	
	Note: double precision for exp(x) was not required.
	
	\end{verbatim}
	I chose $h = 40.0$ for $cos(x)$ and $\exp(x)$ and $h=0.1$ for $\sqrt{x}$\\
	The step size was reduced upto the machine precison upto $1e-6$ for single precision
	and $1e-14$ for double precision.
	
	
	\subsection{part b}
	The source code is:\\
	hw4qn1.f90\\
	
	output dat files are:\\
	cos01sp.dat,cos01dp.dat  etc.\\
	I plotted modulus of logE vs logH  for 3 functions for single and double precisions.\\
	The number of decimal places obtained agrees with the estimates in the text.\\
	The graphs looks like this:\\
	
	%%%%% including figure %%%%%%%%%%%%%%%%%%
	\begin{figure}[h!]
	\centering
	\includegraphics [scale=0.6]{cos01sp.eps}
	\caption{Plot of $ \cos(0.1)$ single precision }
	\end{figure}
	%%%%%%%%%%%%%%%%%%%%%%%%%%%%%%%%%%%%%%%%
	
	%%%%% including figure %%%%%%%%%%%%%%%%%%
	\begin{figure}[h!]
	\centering
	\includegraphics [scale=0.6]{cos01dp.eps}
	\caption{Plot of $ \cos(0.1) $ double precision }
	\end{figure}
	\clearpage
	%%%%%%%%%%%%%%%%%%%%%%%%%%%%%%%%%%%%%%%%
	
	\clearpage
	%%%%% including figure %%%%%%%%%%%%%%%%%%
	\begin{figure}[h!]
	\centering
	\includegraphics [scale=0.6]{exp01sp.eps}
	\caption{Plot of $ \exp(0.1)$ single precision }
	\end{figure}
	\clearpage
	%%%%%%%%%%%%%%%%%%%%%%%%%%%%%%%%%%%%%%%%
	
	
	
	
	
	\subsection{part c}

	
	I plotted the best fit of the graph with log-log fit from xmgrace.
	Appropriate range for X and Y axes were chosen.
	From the plots we know that for cosine function in double precision:\\
	slope for forward differentiation is nearly 1.\\
	slope for central differentiation is nearly 2.\\
	slope for extrapolation differentiation is nearly 4.\\
	
	The source code is:\\
	assign4/qn1/hw4qn1.f90\\
	
	The output plots are:\\
	cos01dp.eps,cos1dp.eps,etc\\
	
	
	
	\subsection{part d}
	
	I repeated the analysis for $\cos(x)$ and $\sqrt{x}$ in double precision and
	compared to the single precision.
	
	%%%%% including figure %%%%%%%%%%%%%%%%%%
	\begin{figure}[h!]
	\centering
	\includegraphics [scale=0.6]{hw4qn2_1.eps}
	\caption{Plot of $ \cos(0.1)$ }
	\end{figure}
	%%%%%%%%%%%%%%%%%%%%%%%%%%%%%%%%%%%%%%%%
	
	
	The source code is:\\
	assign4/qn1/hw4qn1.f90\\
	
	The output data files are:\\
	cos01dp.dat,cos1dp.dat,etc\\
	
	The output plots are:\\
	cos01dp.eps,cos1dp.eps,etc\\
	
	
	
	\subsection{part e}
	In the above plots I paid special attention to the algorithmic errors.
	The best fit was plotted for the algorithmic error part.
	The best fit eps files are inside qn1 folder.
	

	
%%%%%%%%%%%%%%%%%%%%%%%%%%%%%%%%%%%%%%%%%%%%%%%%%%%%%%%%%%%%%%%%%%%%%%%%%%%%%%%%%%%%%%%%%%%%%%%%%%%%%%%%%%
\section{Question 2}
In this question we studied three-point and five point formula for second order derivative
of functions.
\subsection{part a}
The three-point and five-point formula was derived and the pdf can be
found inside:
writeup/hw4qn2a.pdf\\
or, qn1/hw4qn2a.pdf\\


\subsection{part b}
I wrote the code to calculate 2nd order derivative for $cos(x)$ in single precision for the 3 values of 
$x=0.1$,$x=1$,$x=30$. I started with h=0.314 and keep going down upto machine precision 1e-6.\\
While calculating derivative special attention was given in grouping the terms. Similar terms are grouped
together.\\
The grouping can be seen in the source code.\\
The source code is:\\
qn2/hw4qn2.f90\\

and the outpuf files are:\\
$hw4qn2\_1.dat, hw4qn2\_01.dat,and, hw4qn2_30.dat.$\\

\subsection{part c}
The derivative and its relative errors were produced.
I reduced step size upto 1e-6 for single precision.

\subsection{part d}
The log-log plot of logE versus logH was created.
The plots are:\\
$hw4qn2\_1.eps, hw4qn2\_01.eps,and, hw4qn2_30.eps.$\\
The number of decimal places obtained agrees with the estimates in the text.

\subsection{part e}
We can see truncation error at large h and roundoff error at small h in the graph.


\section{Question 3: Population Growth Problem}
In this problem we solved population growth equation both numerically and analytically.
\subsection{part a: when b=0}
First we took, $b=0$ and solved the equation.
The source code is:\\
$qn3/hw4qn3a.f90$\\

The output is:\\
$qn3/hw4qn3a.dat$\\

The plot is:\\
$qn3/hw4qn3a.eps$\\
\clearpage

\subsection{part b: when b=3}
Here, we took a =10 and b=3.
The source code is:\\
qn3/hw4qn3b.f90\\

The output is:\\
qn3/hw4qn3b.dat\\

the plot is :\\
qn3/hw4qn3b.eps\\


Initially the population decreases with time since the $-bN^2$ term dominates and as the 
time passes by population growth seems to be constant. We can see this in the plots.



\end{document}

