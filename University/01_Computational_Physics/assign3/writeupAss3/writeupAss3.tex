% cmd  : clear; latex writeupAss3.tex
% cmd  : clear; xdvi writeupAss3.dvi
% cmd  : clear; dvipdf writeupAss3.dvi


\documentclass[11pt,a4paper,english]{article}
\usepackage{babel}
\usepackage{amssymb}
\usepackage{graphicx,subfigure}
\usepackage[export]{adjustbox}    % for positioning figure
\usepackage{textcomp}
\usepackage{fixltx2e}
\usepackage[usenames,dvipsnames,svgnames,table]{xcolor}

%some useful newcommands
\newcommand{\beq}{\begin{equation}}
\newcommand{\eeq}{\end{equation}}
\newcommand{\bfig}{\begin{figure}}
\newcommand{\efig}{\end{figure}}
\newcommand{\beqa}{\begin{eqnarray}}
\newcommand{\eeqa}{\end{eqnarray}}
\newcommand{\beqan}{\begin{eqnarray*}}
\newcommand{\eeqan}{\end{eqnarray*}}
\newcommand{\ba}{\begin{array}}
\newcommand{\ea}{\end{array}}
\newcommand{\ben}{\begin{enumerate}}
\newcommand{\een}{\end{enumerate}}
\newcommand{\bfl}{\begin{flushleft}}
\newcommand{\efl}{\end{flushleft}}
\newcommand{\btab}{\begin{tabular}}
\newcommand{\etab}{\end{tabular}}
\newcommand{\bit}{\begin{itemize}}
\newcommand{\eit}{\end{itemize}}
\newcommand{\bdes}{\begin{description}}
\newcommand{\edes}{\end{description}}
\newcommand{\bdm}{\begin{displaymath}}
\newcommand{\edm}{\end{displaymath}}
\newcommand {\IR} [1]{\textcolor{red}{#1}}

% for listing
\usepackage{enumitem}
\usepackage[ampersand]{easylist}
\ListProperties(Hide=100, Hang=true, Progressive=3ex, Style*=-- ,
Style2*=$\bullet$ ,Style3*=$\circ$ ,Style4*=\tiny$\blacksquare$ )    % for easylist
\newcommand{\begl}{\begin{easylist}}
\newcommand{\eegl}{\end{easylist}}

% for hyperlink
\usepackage{hyperref}             % for hyperlink
\hypersetup{
    colorlinks=true,
    linkcolor=blue,
    filecolor=magenta,      
    urlcolor=cyan,
    pdftitle={Sharelatex Example},
    bookmarks=true,
    pdfpagemode=FullScreen,
}


% Creating Title for the assessment

\title{Assignment 3}
\author{Bhishan Poudel}
\date{Sep, 2015}

% begin of document
\begin{document}
\maketitle
\tableofcontents
\listoffigures
\clearpage

\section{Finding roots}

	This program computes the root of the equation
	\beqa
	f(x)=x^2 - 7x -ln(x)
	\eeqa
	using bisection and secant method. \\
	
	\subsection{Bisection Method}
	\subsubsection{first root}
	
	The source code is:\\
	assign3/qn1/bisection.f90\\
	
	to comile and run this code for the first root:\\
	%f90 bisection.f90 && ./a.out > bisection1.dat
	
	f90 bisection.f90 \&\& ./a.out $>$ bisection1.dat
	
	then, we can see output file in the path:\\
	assign3/qn1/bisection1.dat\\
	
	note that: at the line 14, the code fragment looks like
	\begin{verbatim}
	!! initial range (x0,x1) to find the root within it

    x0 = 0.1d0   !! for first root
    x1 = 1.0d0   
    !x0 = 5.1d0   !! for second root
    !x1 = 10.0d0
	\end{verbatim}
	
	\subsubsection{second root}
	
	The source code is:\\
	assign3/qn1/bisection.f90\\
	
	to comile and run this code for the second root:\\
	%f90 bisection.f90 && ./a.out > bisection2.dat
	
	f90 bisection.f90 \&\& ./a.out $>$ bisection2.dat
	
	then, we can see output file in the path:\\
	assign3/qn1/bisection2.dat\\
	
	note that: at the line 14, the code fragment looks like
	\begin{verbatim}
	!! initial range (x0,x1) to find the root within it

    !x0 = 0.1d0   !! for first root
    !x1 = 1.0d0   
    x0 = 5.1d0   !! for second root
    x1 = 10.0d0
	\end{verbatim}
	%%%%%%%%%%%%%%%%%%%%%%%%%%%%%%%%%%%%%%%%%%%%%%%%%%%%%%%%%%%%%%%%%%%%%%%%%%%%%%%%%%%%%%%%%%%%%%%
	
	\subsection{Newton-Raphson Secant Method}
	\subsubsection{first root}
	
	The source code is:\\
	assign3/qn1/secant.f90\\
	
	to comile and run this code for the first root:\\
	%f90 secant.f90 && ./a.out > secant1.dat
	
	f90 secant.f90 \&\& ./a.out $>$ secant1.dat
	
	then, we can see output file in the path:\\
	assign3/qn1/secant1.dat\\
	
	note that: at the line 33, the code fragment looks like
	\begin{verbatim}
	!FIRST ROOT
      xinitial = 0.1d0  
      xfinal   = 1.0d0
       
      ! SECOND ROOT
      !xinitial = 5.1d0  
      !xfinal   = 10.0d0 
	\end{verbatim}
	
	\subsubsection{second root}
	
	The source code is:\\
	assign3/qn1/secant.f90\\
	
	to comile and run this code for the second root:\\
	%f90 secant.f90 && ./a.out > secant2.dat
	
	f90 secant.f90 \&\& ./a.out $>$ secant2.dat
	
	then, we can see output file in the path:\\
	assign3/qn1/secant2.dat\\
	
	note that: at the line 33, the code fragment looks like
	\begin{verbatim}
	!FIRST ROOT
      !xinitial = 0.1d0  
      !xfinal   = 1.0d0
       
      ! SECOND ROOT
      xinitial = 5.1d0  
      xfinal   = 10.0d0 
	\end{verbatim}
	
	
	
	
	
	%%%%%%%%%%%%%%%%%%%%%%%%%%%%%%%%%%%%%%%%%%%%%%%%%%%%%%%%%%%%%%%%%%%%%%%%%%%%%%%%%%%%%%%%%%%%%%%
	\subsection{Comparison}
	\begl
	& The positive roots of the equation was found upto five significant figures.\\
	& In comparison to secant method the bisection method is slower.\\
	  For example: \\
	  To find the the first root 0.22203, bisection method required 18 iterations and secant method
	  needed 6 iterations.
	  We can see that difference in:\\
	  assign3/qn1/bisection1.dat\\
	  assign3/qn1/secant1.dat\\
	  
	& While using these Bisection and Secant method we need prior approximate position of zero.
	  
	  
	
	\eegl
	

	
	
\clearpage
\section{Bond Length of the NaCl Molecule}
\subsection{Plot of V(r)}

In this question I wrote a code to plot r vs. V(r).\\
The source code and outputs are:\\
assign3/qn2/ass3qn2.f90\\
assign3/qn2/potentialplot.dat\\
assign3/qn2/potential.eps\\

 	
 	%%%%% including figure %%%%%%%%%%%%%%%%%%
	\begin{figure}[h!]
	\centering
	\includegraphics[width=1.0\textwidth,left]{./images/potential.eps}
	\caption{Plot of r vs. V(r) }
	\end{figure}
	%%%%%%%%%%%%%%%%%%%%%%%%%%%%%%%%%%%%%%%%%
 	
\clearpage

\subsection{Minimum of V(r) and roots}
In this question I wrote a code to find roots when V(r) is minimum.\\
The source code and outputs are:\\
assign3/qn2/ass3qn2.f90\\
assign3/qn2/findroot.dat\\

In the problem we are asked to find equilibrium distance between Na and Cl ions.
After finding the root upto 3 significant figures, I found that 
\beqa r_{eq} = 2.32
\eeqa
To find the root I used secant method, because it is faster than bisection method.
To find root we have also to guess the approximate position of root. First I plotted 
the graph in the website Wolfram Alpha and saw the nature of the graph.
Then I wrote the code. The constant values were provided in the question.


\end{document}

